\appendix

\section{Rule parsing}\label{rule parsing}
Below is a description of our fuzzy rule format, each rule described in this
format should be possible to parse with our current implementation.

All rules have to be on a single line for the parsing to work. This is a flaw
with the parsing mechanisms used and not the runtime structures created.

\lstset{frame=single, breaklines=true}
\begin{lstlisting}[label=lst: bnf, caption=BNF of our rules]
IF		= if COND_ST then action is ACTION
COND_ST		= (COND_ST [and|or] COND_ST) | (FUZZY_EXPR)
FUZZY_EXPR	= LINGVAR [is|not] VAR_NAME
ACTION		= VAR_NAME
LINGVAR		= [a-z]+
VAR_NAME	= [a-z]+
\end{lstlisting}

\section{Example rules}\label{example rules}

\begin{lstlisting}[label=lst:example rules, caption=Example rules representing
project risks]
define lingvar funding: inadequate, marginal, adequate
define lingvar staffing: small, large
define lingvar risk: low, normal, high

define fuzzyset funding.inadequate: grade = [(28, 1), (43, 0)]
define fuzzyset funding.marginal: triangle = [(28, 0), (70, 1), (112, 0)]
define fuzzyset funding.adequate: grade = [(84, 0), (112, 1)]

define fuzzyset staffing.small: grade = [(33, 1), (64, 0)]
define fuzzyset staffing.large: grade = [(44, 0), (66, 1)]

define fuzzyset risk.low: grade = [(20, 1), (40, 0)]
define fuzzyset risk.normal: triangle = [(20, 0), (50, 1), (80, 0)]
define fuzzyset risk.high: grade = [(60, 0), (80, 1)]

if ((funding is adequate) or (staffing is small)) then risk is low
if ((funding is marginal) and (staffing is large)) then risk is normal
if (funding is inadequate) then risk is high
\end{lstlisting}
