\section{Fuzzy logic}\label{fuzzy logic}
In this section we will describe how our fuzzy logic was implemented and try to
give some arguments for some of the decisions that we made to arrive
at this point.

Our fuzzy logic is divided into three parts. We have the rule parsing which, as
the name implies, parses a specific set of rules and converts them into a runtime
structure. The runtime structure is tasked with representing the fuzzy logic and
letting other parts of our code interact to get results back. The runtime structure
consists of three parts, the "if-statement", the "conditional statement" and
the "fuzzy expression". The fuzzy expression consists internally of fuzzy sets
which contains the knowledge about how rules should be divided into sets of truth
values. The last part of the fuzzy logic is the "reasoner" which contains two 
methods for inference, Mamdani and Sugeno.

\subsection{Runtime structure}\label{runtime structure}
As stated, the runtime structure consists of three parts, each corresponding to
a different statement in our rule parsing(see \ref{rule parsing}). To be able
to parse as diverse rules as possible we designed the runtime structure around
the notion of generality. Each rule is broken down into separate pieces and for
each such piece we create a separate runtime structure to keep track of what
is going on. When we evaluate a rule we propagate the necessary information
down trough all the structures before we are left with nothing but fuzzy expressions.

The top layer of the structure is the if-statements, these represent individual
rules which lead to a result. The if-statement is what the outside world sees
of our structure and it is the only place to interact with. The if-statement
consists of conditional expressions which hold either a fuzzy expression or
a concatenation of conditional expressions. The bottom layer of the structure
is the fuzzy expressions, which holds the variables and the fuzzy sets to evaluate
those variables against. The fuzzy sets are structures which can evaluate whether
a value is within their range and, if so, to what degree. For this project
we support sets representing triangles, trapezoids and simple gradients.

In appendix \ref{example rules} we have an example of what the runtime structure
has to support. The "define fuzzyset" lines defines fuzzy sets as described above.
The lines beginning with "if" represents a complete if statement. An expression
surrounded by parentheses represents either a fuzzy expression, or if there is an
"and" between to such expressions, a conditional statement.

\subsection{Fuzzy Inference}\label{fuzzy inference}
To make use of the runtime structure we have to implement some sort of inference
which can use the rules created to give output in some form. This is done by
doing inference on the input. In our cases we get input from the simulator
telling us about the variables in the system. Then we use either Mamdani or
Sugeno inference to interpret those variables and what they mean. Our inference
returns the action most associated with the rules, or if no rules fires at all
we throw an exception which must be dealt with other places.

For Mamdani inference we go over a certain number of steps in the output space
and sum up all the values and their degree from all the rules. To find out
whether or not a certain value is within a certain set we decided that we would
try all of them and select the one which has the highest degree of "ownership"
of that value. This is slow, but it is easy to think about and easy to implement.

Since Sugeno inference is slightly easier, all we have to do is to sum up the middle
value from each result we get. This is much faster, but it does lack the resolution
of Mamdani. With the simple rules that we have tried for our mice, Mamdani works
fine, but if this would become a problem we could easily switch to Sugeno.
