\section{Introduction}\label{intro}
Fuzzy logic is a type of logic where each value can take on a range of truth values.
This type of logic deals with degrees of truth in the rules it forms, each expression
in the language can take on many values and all rules might fire to some degree.

Fuzzy logic then is well suited for situations where we want to use expressions that
a human might use. For example, in first order logic we can express things like "The
car is traveling at a speed over 50km/h", this is all well and good, but in most regular
conversations humans would not express them self like this. Instead humans would use
a sentence like "The car is traveling quite fast". This sentence is hard to represent
in first order logic since we have a couple of variables which does not have a single 
value. "Fast" is not something a computer can work with straight out of the box since 
"Fast" might be a range of values. This is where fuzzy logic comes in. In fuzzy logic
we can explain "Fast" as a range of values and when asked if the car is traveling "fast"
we can evaluate that to a degree of truth. The car might not be traveling that fast and
so our rules might say that it is fast to a degree of "0.2".

In this project we have implemented fuzzy logic with the aim of controlling mice which
are running around a designated area. We read in rules written in a special format (see
\ref{rule parsing}) and then use those rules to control the mice with fuzzy logic.

We will first start talking a bit about our fuzzy logic implementation. Then we will move
over to the Qt simulation. Finally we will wrap up with some thoughts and results.
