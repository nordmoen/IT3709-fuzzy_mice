\section{Qt Simulation}\label{qt simulation}
The Qt\footnote{\url{http://qt-project.org/}} simulation is not necessarily the
meat of our project, but it's the part that glues it all together and helps us
visualize how the inference is working.

Most of the simulation is based on the colliding mice example created by Qt
for its C++ UI framework. We have used the PySide\footnote{\url{http://qt-project.org/wiki/PySide}}
bindings and used the colliding mice example\footnote{\url{http://qt.gitorious.org/pyside/pyside-examples/blobs/bc97d0b794dfd153462ac409569d73dd991b4e1f/examples/graphicsview/collidingmice/collidingmice.py}}
to base most of our simulation.

We have extended the example with fuzzy logic and given each mouse health, strength and speed
which controls different aspects of the mice and the fuzzy logic.

From the code that we reused from Qt each mice performs one action three times a
second. In this time each mouse tries to stay within a certain range of the middle,
but are controlled by our fuzzy logic, if any of the rules fire. Each mouse
has a "fuzzy reasoner" which deals with the inference. This reasoner has been created
with the interpreted fuzzy rules and takes a certain number of arguments. We then
identify the two strongest mice within view and feed the reasoner the values associated
with the other mice and our own health. The reasoner then uses the fuzzy rules and
evaluates which action should be taken. With these actions in hand the mice locates
either the most dangerous of the two enemies or the weakest and then goes to work.
