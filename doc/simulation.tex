\section{Qt Simulation}\label{qt simulation}
The Qt\footnote{\url{http://qt-project.org/}} simulation is not necessarily the
meat of our project, but it's the part that glues it all together and helps us
visualize how the inference is working.

Most of the simulation is based on the colliding mice example\footnote{\url{http://qt.gitorious.org/pyside/pyside-examples/blobs/bc97d0b794dfd153462ac409569d73dd991b4e1f/examples/graphicsview/collidingmice/collidingmice.py}}
created by Qt for its C++ UI framework. In combination with the PySide\footnote{\url{http://qt-project.org/wiki/PySide}}
bindings, that allowed us to use python with Qt, we had a simple framework to work with.

The example features several mice running around on a small cheese-field, turning around when they
reach the edge. We have extended it with fuzzy logic and given each mouse health, strength and speed,
to control different aspects of the mice and the fuzzy logic. Whenever two or more mice collide with each
other, a fight will occur, and damages are dealt semi-randomly based on the strength of each mouse.
In order to restrict the number of fights per second, we added a counter that only allows a mouse to fight
every fifth timer-tick.

From the code that we reused from Qt, each mouse will perform an action 30 times per
second (specifically, we use a timer on 1000/33ms intervals). In this time, each mouse tries to stay within
a certain distance to the middle of the field, but are controlled by our fuzzy logic if any of the rules fire.
Each mouse has a "fuzzy reasoner" which deals with the inference. This reasoner has been created
with the interpreted fuzzy rules and takes a certain number of arguments. We then
identify the two strongest mice within view and feed the reasoner the values associated
with the other mice and our own health. The reasoner then uses the fuzzy rules and
evaluates which action should be taken. This gives us one action per mouse, 
which are then prioritized by type: fleeing is more important than attacking,
and attacking is more important than ignoring. Knowing which mouse to react to, we
then steer the mouse towards, or away from the most important mouse we identified.
If we decide to ignore both mice, we just continue straight ahead. In the case where nothing
is returned from the reasoner, we default to a fake ignore-action, as it is usually
caused by having no mice in view.

To easier visualize what each mouse is doing, we made them change ear-color according to the current action:
dark red for attacking, gray for fleeing, and dark yellow when ignoring. When two mice collide, their ears
turn bright red.